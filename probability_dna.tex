\documentclass[12pt,a4paper]{article}
\usepackage[utf8]{inputenc}
\usepackage[russian]{babel}

\usepackage{amsmath}
\usepackage{amsfonts}
\usepackage{amssymb}
\usepackage{url}
\usepackage[left=2cm,right=2cm,top=2cm,bottom=2cm]{geometry}


\DeclareMathOperator{\Var}{Var}
\DeclareMathOperator{\E}{E}



\title{Теория вероятностей: культурный код}
\author{Фольклор}
\date{\today}
\begin{document}

\maketitle

\section{Культурный код}

Сборник восхитительных задач по элементарной теории вероятностей. Эти задачи --- наша вероятностная ДНК.

\section{Фольклор и красота}


\begin{enumerate}
\item Задача о Сумасшедшей Старушке.

В самолете $100$ мест и все билеты проданы. Первой в очереди на посадку стоит Сумасшедшая Старушка. Сумасшедшая Старушка несмотря на билет садиться на случайно выбираемое место. Каждый оставшийся пассажир садится на своё место, если оно свободно и на случайное выбираемое место, если его место уже кем-то занято.

\begin{enumerate}
% \item Какова вероятность того, что все пассажиры сядут на свои места?
%\item Какова вероятность того, что второй пассажир в очереди сядет на своё место? 
\item Какова вероятность того, что последний пассажир сядет на своё место?
\item Чему примерно равно среднее количество пассажиров севших на свои места?
\end{enumerate}


\item Задача собирателя наклеек. Coupon collector's problem. 

Производитель чудо-юдо-йогуртов наклеивает на каждую упаковку одну из 50 случайно выбираемых наклеек. Покупатель собравший все виды наклеек получает приз от производителя. Пусть $X$ --- это количество упаковок йогурта, которое нужно купить, чтобы собрать все наклейки.

Найдите  $\E(X)$, $\Var(X)$

\item Спички Банаха. Banach's matchbox problem.

Польский математик Стефан Банах имел привычку носить в каждом из двух карманов пальто по коробку спичек. Всякий раз, когда ему хотелось закурить трубку, он выбирал наугад один из коробков и доставал из него спичку. Первоначально в каждом коробке было по $n$ спичек. Но когда-то наступает момент, когда выбранный наугад коробок оказывается пустым.

\begin{enumerate}
\item Какова вероятность того, что в другом коробке в этот момент осталось ровно $k$ спичек?
\item Каково среднее количество спичек в другом коробке?
\end{enumerate}


\item Равновесие Харди-Вайнберга. 

Предположим, что три возможных генотипа \verb|aa|, \verb|Aa| и \verb|AA| изначально встречаются с частотами $p_1$, $p_2$ и $p_3$, где $p_1+p_2+p_3=1$. Ген не сцеплен с полом, поэтому частоты $p_1$, $p_2$ и $p_3$ одинаковы для мужчин и для женщин. 
\begin{enumerate}
\item У семейных пар из этой популяции рождаются дети. Назовём этих детей первым поколением. Каковы частоты для трёх возможных генотипов в первом поколении? 
\item У семейных пар первого поколения тоже рождаются дети. Назовём этих детей вторым поколением. Каковы частоты для трёх возможных генотипов во втором поколении? 
\item Каковы частоты для трёх возможных генотипов в $n$-ном поколении?
\item Заметив явную особенность предыдущего ответа сформулируйте теорему о равновесии Харди-Вайнберга. Прокомментируйте утверждение: <<Любой рецессивный ген со временем исчезнет>>.
\end{enumerate}

\item Поляризация света.

Световая волна может быть разложена на две поляризованные составляющие, вертикальную и горизонтальную. Поэтому состояние отдельного поляризованного фотона может быть описано\footnote{На самом деле внутренний мир фотона гораздо разнообразнее.} углом $\alpha$. Поляризационный фильтр описывается углом поворота $\theta$. Фотон в состоянии $\alpha$ задерживается поляризационным фильтром с параметром $\theta$ с вероятностью $p=\sin^2(\alpha-\theta)$ или проходит сквозь фильтр с вероятностью $1-p$, переходя при этом в состояние $\theta$. 

\begin{enumerate}
\item Какова вероятность того, что поляризованный фотон в состоянии $\alpha$ пройдёт сквозь фильтр с параметром $\theta=0$?
\item Имеется два фильтра и поляризованный фотон в состоянии $\alpha$. Первый фильтр --- с $\theta=0$, второй --- c $\theta=\pi/2$. Какова вероятность того, что фотон пройдет через оба фильтра?
\item Имеется три фильтра и поляризованный фотон в состоянии $\alpha$. Первый фильтр --- с $\theta=0$, второй --- c $\theta=\beta$, третий --- с $\theta=\pi/2$. Какова вероятность того, что фотон пройдет через все три фильтра? При каких $\alpha$ и $\beta$ она будет максимальной и чему при этом она будет равна?
\item Объясните следующий фокус. Фокусник берет два специальных стекла и видно, что свет сквозь них не проходит. Фокусник ставит между двумя стёклами третье, и свет начинает проходить через три стекла. 
\end{enumerate}

Тут ссылки на видео и <<секретный монитор>>

\item Истеричная певица

Начинающая певица дает концерты каждый день. Каждый ее концерт приносит продюсеру 0.75 тысяч евро. После каждого концерта певица может впасть в депрессию с вероятностью 0.5. Самостоятельно выйти из депрессии певица не может. В депрессии она не в состоянии проводить концерты. Помочь ей могут только цветы от продюсера. Если подарить цветы на сумму $0\le x\le 1$ тысяч евро, то она выйдет из депрессии с вероятностью $\sqrt{x}$. 

Какова оптимальная стратегия продюсера? 

\item Парадокс Симпсона.  Simpson's Paradox.

Два лекарства испытывали на мужчинах и женщинах. Каждый
человек принимал только одно лекарство. Общий процент людей,
почувствовавших улучшение, больше среди принимавших лекарство А.
Процент мужчин, почувствовавших улучшение, больше среди мужчин, принимавших лекарство В. Процент женщин, почувствовавших улучшение, больше среди женщин, принимавших лекарство В. 

Возможно ли это? 


\end{enumerate}

\section{Опорные задачи}

\begin{enumerate}
\item Стрелок попадает по мишение с вероятностью 0.3. Какова вероятность того, что до третьего промаха у него будет 5 выстрелов?

\item Расстояние от пункта A до B автобус проходит за 2 мин, а пешеход — за 20 мин. Интервал движения автобусов 30 мин. Вы подходите в случайный момент времени к пункту A и отправляетесь в B пешком. 

\begin{enumerate}
\item Какова вероятность того, что в пути Вас догонит очередной автобус?
\item Сколько в среднем времени Вы будете добираться, если проходящий мимо автобус обязательно Вас подбирает?
\end{enumerate}

% 18/30, примерно 14 минут

\end{enumerate}

\section{Черновые размышления :)}

Сумасшедшая старушка  ок

Спички Банаха  ок

Наклейки ок

Дон-Жуан

Разборчивая невеста (\url{http://en.wikipedia.org/wiki/Secretary_problem})

Два конверта

Монти-Холл

Обезьяны и Шекспир (Абракадабра)

Двойное зеро на рулетке (вероятность выигрыша, сравнить с одним зеро)

ООР или РОО

Триэль

Две выигрышных игры

Парадок инспектора

Гороскоп и пол ребенка

Гадалка и выбор наибольшего числа

Задача о спящей красавице


Нестандартные задачи ??? ()

Автобус догоняет по дороге (вероятность и среднее) ok

певица и депрессия ok

удвоение ставки - броуновское движение


Методы: ???

Первый шаг

Разложение в сумму - Разделяй и властвуй

Большая сила о-малых

Матрешки - (условное математическое ожидание)

Холмс, это невероятно! (Вероятностная интерпретация)

Как сварить мартингал (теорема Дуба)

Свернулся колечком (задачи на равномерное на отрезке)


Сюжеты: ????

Пуассоновский поток

Броуновское движение



\end{document}